\problemname{Secret Sequence}
There is a secret sequence of zeros and ones with length $n$, and you want to know the number of ones in the sequence.
You are allowed to ask queries by giving four integers $0 \leq a \leq b \leq c \leq d \leq n$. The answer
to the query will be $-1$ if the sum of the numbers at positions $a, a+1, ..., b-1$ is larger than the
sum of the numbers at positions $c, c+1, ..., d-1$, and $1$ if the sum is smaller. If the sums are equal,
the answer will be $0$.

The indices of the sequence start at $0$ and end at $n-1$. Note that the intervals you're querying can be empty,
if $a = b$ or $c = d$ respectively. The sum of the numbers in an empty interval is $0$.

Figure out the total number of ones in the sequence using as few queries as possible.

\section*{Interactivity}
This problem is interactive.

Your program should start by reading an integer $n$, the length of the sequence.

Then for each query you make, you should print a single line containing a question mark (?) and
four integers $0 \leq a \leq b \leq c \leq d \leq n$, separated by spaces.
The grader will then read these numbers, and in return print a line with either $-1$, $0$ or $1$, which can be read by your program.

Your program should then repeat this until you want to guess how many ones there are,
which you do by printing a line containing an exclamation mark (!) followed by a single integer number $x$, separated by a space.
After you printed this line, your program should terminate (otherwise it might be judged as Time Limit Exceeded).
If the number $x$ is correct, i.e if it is the number of ones in the sequence, you pass the test case.

You \emph{must} make sure to flush standard output before reading the grader's response, or else your program
will get judged as Time Limit Exceeded. This works as follows in various languages:
\begin{itemize}
  \item Java: \texttt{System.out.println()} flushes automatically.
  \item Python: \texttt{print()} flushes automatically.
  \item C++: \texttt{cout << endl;} flushes, in addition to writing a newline. If using printf, \texttt{fflush(stdout)}.
  \item Pascal: \texttt{Flush(Output)}.
\end{itemize}

\section*{Constraints}
The number $n$ is at most $2 \cdot 10^5$. You can make at most $200$ queries.
Your solution will be tested on a set of test groups, each worth a number of points.
Each test group contains a set of test cases.
To get the points for a test group you need to solve all test cases in the test group.
Your final score will be the maximum score of a single submission.

\noindent
\begin{tabular}{| l | l | l |}
\hline
Group & Points & Constraints \\ \hline
1     & 5     & There are either no ones or a single one. \\ \hline
2     & 8    & $n \leq 200$ \\ \hline
3     & 10    & The ones form a continuous segment. \\ \hline
4     & 12    & $n \leq 4000$ \\ \hline
5     & 35    & No constraints. \\ \hline
\end{tabular}

\section*{Explanation of Sample 1}
The secret sequence in this sample is $01101$, so $n = 5$, which is what the judge writes on the first line.
The program then asks which number is greater: the first (at position $0$) or the last (at position $4$).
The judge answers that the last number is greater, so we then know it has to be a one while the first number is a zero.
After this the program asks which is greater: the sum of the numbers at position $1$ and $2$ or the number at position $4$.
The judge answers that the sum of the two numbers at position $1$ and $2$ is greater, and since we already
know that there is a one at position $4$, this means both the numbers at position $1$ and $2$ are ones.
Finally the program asks which is greater: the number at position $0$ or the number at position $3$.
The judge answers that they are the same, and so since the number at position $0$ is zero, which we know
from the first query, this is also the case for the number at position $3$.

Hence there are $3$ ones in total, so the program answers this.
