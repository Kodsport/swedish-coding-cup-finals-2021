\problemname{Hemligfoljd}
TODO: SVENSK VERSION

There is a secret sequence of zeros and ones with length $n$, and you want to know the number of ones in the sequence.
You are allowed to ask queries by giving four integers $0 \leq a \leq b \leq c \leq d \leq n$. The answer
to the query will be $-1$ if the sum of the numbers in the interval $[a,b)$ is larger than the sum of the numbers
in the interval $[c,d)$, and $1$ if the sum is smaller. If the sums are equal, the answer will be $0$.

Figure out the total number of ones in the sequence using as few queries as possible.

\section*{Interactivity}
This problem is interactive.

Your program should start by reading an integer $n$, the length of the sequence.

Then for each query you make, you should print a single line containing a question mark ("?") and
four integers $0 \leq a \leq b \leq c \leq d \leq n$, separated by spaces.
The grader will then read these numbers, and in return print a line with either $-1$, $0$ or $1$, which can be read by your program.

You should then repeat this until you want to guess how many ones there are,
which you do by printing a line containing an exclamation mark ("!") followed by a single number $x$, separated by a space.
After you printed this line, your program should terminate (otherwise it might be judged as Time Limit Exceeded).
If the number $x$ is correct, you pass the test case.

You \emph{must} make sure to flush standard output before reading the grader's response, or else your program
will get judged as Time Limit Exceeded. This works as follows in various languages:
\begin{itemize}
  \item Java: \texttt{System.out.println()} flushes automatically.
  \item Python: \texttt{print()} flushes automatically.
  \item C++: \texttt{cout << endl;} flushes, in addition to writing a newline. If using printf, \texttt{fflush(stdout)}.
  \item Pascal: \texttt{Flush(Output)}.
\end{itemize}

\section*{Constraints}
The number $n$ is at most $2 \cdot 10^5$.
Your solution will be tested on a set of test groups, each worth a number of points.
Each test group contains a set of test cases.
To get the points for a test group you need to solve all test cases in the test group.
Your final score will be the maximum score of a single submission.

\noindent
\begin{tabular}{| l | l | l |}
\hline
Group & Points & Constraints \\ \hline
1     & 5     & There is either no ones or a single one. \\ \hline
2     & 10    & The ones form a continuous segment. \\ \hline
3     & 20    & You can use at most $n$ queries. \\ \hline
4     & 35    & You can use at most $350$ queries. \\ \hline
5     & ?     & Ones form a continuous segment and a 3*log(n) solution? \\ \hline
6     & ?     & Konstantoptimering?
\end{tabular}
