\problemname{Hemligfoljd}   % Hemlig Följd
En hemlig följd innehåller $n$ tal, och varje tal är antingen en etta eller en nolla. Du vill ta reda på hur många
ettor som finns i följden. För att göra det får du ställa frågor genom att specificera fyra heltal
$0 \leq a \leq b \leq c \leq d \leq n$. Svaret på en fråga är $-1$ om summan av talen i intervallet $[a,b)$ är större
än summan av talen i intervallet $[c,d)$, medan den är $1$ om summan är mindre. Om summorna är lika, så blir svaret $0$.
Talen i följden är numrerade från $0$ till $n-1$.

Ta reda på hur många ettor som finns i följden genom att ställa så få frågor som möjligt.

\section*{Interactivity}
Detta problem är interaktivt.

Du ska första läsa in ett heltal $n$, längden på följden.

Sen ska du, för varje fråga du ställer, skriva ut en rad med ett frågetecken ("?") och fyra heltal
$0 \leq a \leq b \leq c \leq d \leq n$, med mellanslag mellan. Domaren kommer läsa dessa tal,
och svara med en rad med ett tal, antingen $-1$, $0$ eller $1$, som ditt program ska läsa.

Ditt program ska upprepa detta tills du vill gissa hur många ettor som finns i följden. Det gör
du genom att skriva ut en rad med ett utropstecken ("!") följt av ett enda heltal $x$, antalet ettor
du tror finns i följden. När du skrivit ut detta tal ska ditt program avslutas (annars kan det bli
Time Limit Exceeded). Om talet $x$ är korrekt, klarar du det testfallet.

Du \emph{måste} se till att flusha standard output innan du läser domarens svar, annars kommer ditt program
få Time Limit Exceeded. Det gör man på följande vis i olika språk:
\begin{itemize}
  \item Java: \texttt{System.out.println()} flushes automatically.
  \item Python: \texttt{print()} flushes automatically.
  \item C++: \texttt{cout << endl;} flushes, in addition to writing a newline. If using printf, \texttt{fflush(stdout)}.
  \item Pascal: \texttt{Flush(Output)}.
\end{itemize}

\section*{Begränsningar}
Talet $n$ är som mest $2 \cdot 10^5$. Du kan ställa som mest $200$ frågor.
Din lösning kommer att testas på en mängd testfallsgrupper.
För att få poäng för en grupp så måste du klara alla testfall i gruppen.
Din slutgiltiga poäng är den maximala poäng som uppnåddes i ett och samma försök.

\noindent
\begin{tabular}{| l | l | l |}
\hline
Testgrupp & Poäng & Begränsningar \\ \hline
1     & 5     & Det finns antingen en eller noll ettor. \\ \hline
2     & 10    & $n \leq 200$ \\ \hline
3     & 10    & Alla ettor ligger bredvid varandra. \\ \hline
4     & 45    & Inga ytterligare begränsningar. \\ \hline
\end{tabular}
