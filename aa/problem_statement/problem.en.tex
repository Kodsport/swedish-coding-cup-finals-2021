\problemname{Aa}  % Danish Sorting
The letter Å is a relatively new invention in the Danish alphabet, being introduced only in 1948.
Before that, the digraph Aa was used instead -- this survives in town names like Aabenraa and Aahus.

When sorting Danish words, Å is treated as the last letter of the alphabet.
Interestingly, this partially extends to the digraph: Aa is sorted like Å,
\emph{but only when it represents a single sound}.
Thus, while \texttt{Aahus} (pronunced \texttt{Åhus}) would sort after \texttt{Zurich},
and \texttt{afrikaans} (pronunced \texttt{afrikaans}) after \texttt{afrikan},
\texttt{kontraalt} (pronunced \texttt{kontraalt}) would come before \texttt{kontrabas}.

Given a list of made-up words that could be pronounced in any way, is it possible that it is sorted?

\section*{Input}
The first line of the input contains an integer $N$, the number of words.
The next $N$ lines each contain a non-empty string with characters from \texttt{a-z}, the list of words.

All words will be unique, and there will be at most $500\,000$ characters in total across all words.

\section*{Output}
If it is possible to pick out a set of \texttt{aa}'s in the input words that
should be sorted as Å in such a way that the whole list becomes sorted, output \texttt{yes}.

Otherwise, output \texttt{no}.

\section*{Scoring}
Your solution will be tested on a set of test groups, each worth a number of points.
To get the points for a test group you need to solve all test cases in the test group.
Your final score will be the maximum score of a single submission.

\noindent
\begin{tabular}{| l | l | l | l |}
\hline
Group & Points & Constraints \\ \hline
1     & 7      & $N = 2$ \\ \hline
2     & 10     & $N = 3$ \\ \hline
3     & 10     & $N \le 10$ \\ \hline
4     & 15     & $N \le 10^5$ \\ \hline
\end{tabular}

\section*{Sample Explanations}
TODO
