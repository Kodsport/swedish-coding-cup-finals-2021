\problemname{Ligatures}
Vivi designs fonts. She is almost done with her new masterpiece, but thinks that there is something missing. She has therefore decided to add some ligatures. A ligature occurs when two caracters are combined into a single glyph. For example, "ff" often becomes "ff" and "ae" sometimes becomes "æ". Vivi knows the importance of choosing the right ligature, maybe her ligature will become as lasting as the ligature "&" (which comes from the latin word "et", meaning "or"). Typically, ligatures are chosen for caracter pairs that occur often. A good set of ligatures will not include too many unique ligatures, but will result in many occurences. Taking this task very seriously, she has prepared $Q$ sets of possible ligatures with $K$ unique ligatures in each. She now asks you to test each of her sets of ligatures against her corpus to see which set results in the most occurences. It is important to note that the corpus is read from left to right.

Given a list of suggested sets of ligatures, compute the number of ligature occurrences
across the corpus for each suggestion.

\section*{Input}
The first line of the input contains three integers:
\begin{itemize}
  \item $N$: the size of the corpus ($1 \le N \le 10^6$),
  \item $Q$: the number of suggestions ($1 \le Q \le 10^5$),
  \item $K$: the number of ligatures for each suggestion ($1 \le K \le 20$).
\end{itemize}

The second line of the input contains a string with $N$ characters from \texttt{a-z}: the corpus.

The next $Q$ lines each contain $2K$ characters: the ligature suggestions.
The first two characters make up the first suggested ligature, the second two characters make up the second one, and so on.
No character pair will occur twice within a suggestion.

\section*{Output}
For each suggested ligature set, output a line with the number of ligature occurrences within the corpus,
if the corpus is read from left to right and the given ligatures are applied.

\section*{Scoring}
Your solution will be tested on a set of test groups, each worth a number of points.
To get the points for a test group you need to solve all test cases in the test group.
Your final score will be the maximum score of a single submission.

\noindent
\begin{tabular}{| l | l | l | l |}
\hline
Group & Points & Constraints \\ \hline
1     & 10     & $K = 1$ \\ \hline
2     & 10     & $K \le 2$ \\ \hline
3     & 10     & $K \le 3$ \\ \hline
4     & 10     & $K \le 4$ \\ \hline
5     & 10     & $K \le 5$ \\ \hline
6     & 10     & $K \le 20$ \\ \hline
\end{tabular}

\section*{Explanation of Sample 1}
The ligature "aa" would occur as "ababba \texbft{aa} b \textbf{aa} ab \textbf{aa} \textbf{aa}". It would occur 4 times. 
The ligature "ba" would occur as "a \textbf{ba} b \textbf{ba} \textbf{ba} a \textbf{ba} aa \textbf{ba} aaa". It would occur 5 times. 
Simmilarly, "ab" would occur 5 times, and "bb" would only occur once as "aba \textbf{bb} abaabaaabaaaa".

\section*{Explanation of Sample 2}
For the first set, we have the ligatures "aa" and "ab". Reading the corpus from left to right, we get the 8 occurences: 
"bc \textbf{ab} \textbf{ab} b \textbf{ab} \textbf{aa} b \textbf{aa} \textbf{ab} \textbf{aa} \textbf{aa}".
For the second, "aa" occurs 4 times, and "bb" 1 time, resulting in 5 total occurences.
For the third, "ab" occurs 5 times. "bb" will never occur.
For the fourth set, we have the ligatures "ab" and "ba". Reading the corpus from left to right, we get the 6 occurences: 
"bc \textbf{ab} \textbf{ab} \textbf{ba} \textbf{ba} \textbf{ab} aa \textbf{ab} aaaa".
In a similar fashion, "bc" will occur once in the fith set, and in the last set, "ca" occurs once and "ba" 5 times. 
