\makeatletter

%%
%% Cover page
%%
\newcommand{\coverpage}{
  \pagestyle{empty}
  \begin{center}
    \sf
    {\Huge
      \ifx \@contestname \@empty
      \ifx \@contestshortname \@empty
      \ClassError{problemset}{Must provide either contest name or contest short name for cover page.}{}
      \else
      \@contestshortname
      \fi
      \else
      \@contestname
      \fi
    }\\
    \vspace{5mm}
    %
    \ifx \@contestname \@empty \relax \else
    \ifx \@contestshortname \@empty \relax \else
    {\LARGE \textsl{\@contestshortname}}\\
    \vspace{5mm}
    \fi
    \fi
    \vspace{5mm}
    %
    \ifx \@location \@empty \relax \else
    {\huge \@location}\\
    \vspace{8mm}
    \fi
    %
    \ifx \@contestlogo \@empty \relax \else
    \includegraphics[height=6cm]{\@contestlogo}\\
    \vspace{8mm}
    \fi
    %
    \Huge{\contentsname}\\
    \vspace{7.5mm}
    \LARGE
    \@starttoc{toc}\\
    \vspace{15mm}
    {
      \Large\sf
      Do not open before the contest has started.
    }
  \end{center}
%  \advicepage
  \cleardoublepage
  \pagestyle{fancy}
  \setcounter{page}{1}
}

\newcommand{\advicepage}{

  \clearpage

  \thispagestyle{empty}

  {\Large Advice, hints, and general information}

  \begin{itemize}
  \item
    The problems are not sorted by difficulty.

  \item
    If you think some problem is ambiguous or underspecified, you may
    ask the judges for a clarification request through the Kattis
    system.  The most likely response is ``No comment, read problem
    statement'', indicating that the answer can be deduced by
    carefully reading the problem statement or by checking the sample
    test cases given in the problem.

  \item
    Your submissions will be run multiple times, on several different
    input files.  If your submission is incorrect, the error message you
    get will be the error exhibited on the first input file on which you
    failed.  E.g., if your instance is prone to crash but also incorrect,
    your submission may be judged as either ``wrong answer'' or ``run
    time error'', depending on which is discovered first.


  %% \item
  %%   For problems with floating point output, we only require
  %%   that your output is correct up to some error tolerance.

  %%   For example, if the problem requires the output to be within
  %%   either absolute or relative error of $10^{-4}$, this means that
  %%   \begin{itemize}
  %%   \item If the correct answer is $0.05$, any answer between $0.0499$ and $.0501$ will be accepted.
  %%   \item If the correct answer is $500$, any answer between $499.95$ and $500.05$ will be accepted.
  %%   \end{itemize}
  %%   Any reasonable format for floating point numbers is
  %%   acceptable.  For instance, ``17.000000'', ``0.17e2'', and
  %%   ``17'' are all acceptable ways of formatting the number $17$.  For
  %%   the definition of reasonable, please use your common sense.
\end{itemize}

}

\makeatother
